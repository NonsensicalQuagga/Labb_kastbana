\documentclass[11p]{article}
% Packages
\usepackage{amsmath}
\usepackage{graphicx}
\usepackage[swedish]{babel}
\usepackage[
    backend=biber,
    style=authoryear-ibid,
    sorting=ynt
]{biblatex}
\usepackage[utf8]{inputenc}
\usepackage[T1]{fontenc}
\usepackage{textcomp}
\usepackage{hyperref}
%Källor
\addbibresource{mall.bib}
\graphicspath{ {./images/} }

\title{Labboration kaströrelse \\ \small Fysik 1}
\author{Alvin Högdal}
\date{\today}

\begin{document}

    \begin{titlepage}
        \begin{center}
            \vspace*{1cm}

            \Huge
            \textbf{Labboration kaströrelse}

            \vspace{0.5cm}
            \LARGE
            Fysik 2

            \vspace{1.5cm}

            \textbf{Alvin Högdal}

            \vfill


            \vspace{0.8cm}

            \includegraphics[width=0.4\textwidth]{../images/NTI Gymnasiet_Symbol_print_svart.png}

            \Large
            Teknikprogrammet\\
            NTI Gymnasiet\\
            Umeå\\
            \today

        \end{center}
    \end{titlepage}
% Om arbetet är långt har det en innehållsförteckning, annars kan den utelämnas
    \tableofcontents
    \newpage
    \section{Syfte och frågeställning}
    Att beräkna utgångshastigheten för kulkanonen och därefter kunna beräkna vart kulan kommer landa på bordet och på golvet
    \section{Material och metod}

    \subsection{Material:}
    rada upp dina frågor i punktform
    \begin{itemize}
        \item Kulkanon
        \item Kula
        \item Måttband
        \item kulmål
    \end{itemize}

    \subsection{Metod:}
    Beräkna utgångshastigheten för kulkanonen enligt formeln för stighöjd
    \begin{equation}
        Y_{max} = \frac{v_0^2*\sin^2(\alpha)}{2g}
    \end{equation}
    genom att kolla hut högt kulan far om utgångsvinkeln \alpha \text{ är 90} $^{\circ}$. \\
    Använd sedan det resultatet för att beräkna vart kulan hamnar om utgångsvinkeln \alpha \text{ är 60} $^{\circ}$.
    Beräkna också vart kulan hamnar om den träffar golvet.
    \paragraph{}

    \section{Resultat}
    Kulan flyger 129 cm upp i luften
    Om man skriver om formeln för stighöjd så att den ger resultatet för hastigheten istället för maxhöjd
    \begin{equation}
       v_0 = \sqrt(\frac{2g*Y_{max}}{\sin^2(\alpha)})
    \end{equation}
    Detta ger resultatet {v_0} = 5,03 m/s.

    Nu kan man beräkna kastvidden enligt formeln för kastvidd
    \begin{equation}
        X_{max} = \frac{v_0^2*\sin^2(\alpha)}{g}
    \end{equation}

    när \alpha \text{ är 60} $^{\circ}$.

    Detta ger resultatet 2.23 m.


    \section{Analys}




\end{document}
